Causal Inference can be described as the process of evaluate cause-and-effect relationships between variables within a system based on observational data. This analysis is crucial in several fields, improves their explainability and reliability and allows us to determine effective solutions in fields such as healthcare, earth science, politics, business, education and many others (\cite{peyrot1996causal}). The information we can obtain thanks to this analysis, for example, the causal factors, are very useful for decision-making of any kind and for predicting the results of potential interventions without actually see them in practice. In fact, while randomized control trials (RCTs) are the gold standard for identifying cause-effect relationships, they are frequently impractical because of high costs and ethical issues. To reach these goals, Causal Inference makes use of Causal discovery algorithms, whose purpose is to uncover the underlying causal structure of a generative processes from a set of observations. These methods rely on specific assumptions and are often represented as a causal graph or a causal adjacency matrix, giving us the possibility to effectively see the relationships between variables and to eventually recognise causes and effects within them (\cite{pearl2009causality}). To illustrate these relationships, causal graphs, often Directed Acyclic Graphs (DAG), use directed arrows, helping us understand the data-generating mechanism more effectively and guiding necessary interventions. Traditional practical applications of predictive systems often overlook causal knowledge, leading to irrational and incomplete decisions when correlations are confused with causation. This can result in significant errors, as highly correlated variables not necessarily influence each other, but could be influenced by hidden factors or latent confounders that have an effect on each of them (\cite{marwala2015causality}).\\

Some of these Causal Discovery methods are designed for static data (non-temporal), while others focus on time series or temporal data. Given that both data types are prevalent in real-world scenarios across different problem domains, it is crucial to have methods capable of recovering causal structures from both types. Time series Causal Discovery methods, more specifically for multivariate scenarios, have been less explored so far, and it's in this field that this thesis tries to give its contribute. The main focus is put on the TD2C method, a quite recent (2015-2024) method that relies on distribution asymmetry brought by causal relationships between variables. In Chapter \ref{Theoretical Background} we are going to display the most important theoretical aspects about Causality and Causal Inference, necessary to understand the procedures we will apply in Chapter \ref{Contributions}. In this second part, a detailed experimental phase is conducted to explore the potentialities of the TD2C method. We will to investigate its applicability to a wide range of possible real-world scenarios and we will try to validate some of its modified versions. In order to fulfill these tests, we will compare TD2C with the most-known state-of-the-art methods for Causal Discovery on multivariate time series.\\

\textit{\textbf{More specifications on what we achieved at the end}}\\

structure tips:\\
causality in generale \\
causal discovery \\
serie temporali\\
i problemi delle serie temporali in causality \\
una breve sintesi delle contribuzioni\\
contesto (dove é avvenuto il lavoro) \\
outline della tesi
